\documentclass[review]{elsarticle}

\usepackage{lineno,hyperref,color}
\modulolinenumbers[5]

\journal{Nuclear Instruments and Methods B}

%%%%%%%%%%%%%%%%%%%%%%%
%% Elsevier bibliography styles
%%%%%%%%%%%%%%%%%%%%%%%
%% To change the style, put a % in front of the second line of the current style and
%% remove the % from the second line of the style you would like to use.
%%%%%%%%%%%%%%%%%%%%%%%

%% Numbered
%\bibliographystyle{model1-num-names}

%% Numbered without titles
%\bibliographystyle{model1a-num-names}

%% Harvard
%\bibliographystyle{model2-names.bst}\biboptions{authoryear}

%% Vancouver numbered
%\usepackage{numcompress}\bibliographystyle{model3-num-names}

%% Vancouver name/year
%\usepackage{numcompress}\bibliographystyle{model4-names}\biboptions{authoryear}

%% APA style
%\bibliographystyle{model5-names}\biboptions{authoryear}

%% AMA style
%\usepackage{numcompress}\bibliographystyle{model6-num-names}

%% `Elsevier LaTeX' style
\bibliographystyle{elsarticle-num}
%%%%%%%%%%%%%%%%%%%%%%%

\begin{document}

\begin{frontmatter}

\title{Fluorescence yield for plastic scintillators after irradiation }


%% or include affiliations in footnotes:
\author[umd]{Zishuo Yang\corref{mycorrespondingauthor}}
\cortext[mycorrespondingauthor]{Corresponding author}
\ead{yangzs@terpmail.umd.edu}
\author[umd]{Alberto Belloni}
\author[umd]{Sarah C. Eno}

\author[eljen]{Charles Hurlbut}
\author[umd]{Geng-Yuan Jeng}
\author[fnal]{Kevin Pedro}
\author[umd]{Yao Yao}
\author[umd]{Adam Zeitlin}



\address[umd]{Dept. Physics, U. Maryland, College Park MD 30742 USA}
\address[eljen]{Eljen Technology, 1300 W. Broadway, Sweetwater, TX 79556 USA}
\address[fnal]{Fermi National Accelerator Laboratory, Batavia, IL, USA}

\begin{abstract}
The ratio of the fluorescence yield versus wavelength for plastic scintillators EJ-200 and EJ-260 before and after irradiation by a $\rm {^{60}Co}$ source for various doses and dose rates and for different concentrations of the primary and secondary dopant was measured.  While the nominal dopant concentration gives the highest light output prior to irradiation, a higher concentration is found to be optimal for irradiated plastics.
\end{abstract}

\begin{keyword}
plastic scintillator\sep fluorescence\sep radiation hardness
\end{keyword}

\end{frontmatter}

\linenumbers

\section{Introduction}
Organic scintillators such as polystyrene (PS) or polyvinyltoluene (PVT)
containing wave-length shifting
additives in solution have long been popular elements in detectors used
in particle physics, nuclear physics, radiation safety, and heath physics applications  due to their high light output, low cost, fast response,
and versatility of physical construction. 
Prolonged exposure of plastic scintillator to
ionizing radiation, however, can result in damage:
light self-absorption (yellowing) increases and
the transfer efficiency of the initial excitation of the polymer to the
dopants combined with the dopant's probability of radiative decay (``initial light output'') can lessen.  
In this paper, we present measurements of the ratio of the light output before and after irradiation
for two different types of plastic scintillator manufactured by Eljen corporation, EJ-200 (similar to BC-408 from Bicron corporation) and EJ-260 (similar to BC-428), before and after irradiation by a $\rm {^{60}Co}$ source for various doses and dose rates and for different concentrations of the primary and secondary dopant.
Dose rate effects are of interest because materials testing is typically done at much higher dose rates than the scintillator will experience in use, because of the cost of reactor time.
Dose rate effects may be responsible for the
factor of three larger light loss with integrated luminosity 
in the CMS endcap calorimeter~\cite{phaseiitdr,ecfa2015}
than expected based
on high dose rate exposures using ${\rm ^{60}Co}$ sources ~\cite{vasken,ByonWagner1993263}.

The effect of dose and dose rate on radiation-induced yellowing have been the subject of many investigations. The results are summarized in ~\cite{sauli}.  
Dose rate effects during irradiation for the yellowing are well documented  ~\cite{sauli,34504,Wick1991472,289295,173180,173178,Giokaris1993315,Wick2001341}.
The presence of oxygen plays an important role~\cite{bross19921199}.  
The studies show that the penetration depth of oxygen into the substrate
depends on the dose rate: at lower dose rates, oxygen penetrates
more deeply~\cite{Wick1991472}.
Because of the importance of the interaction
of oxygen with radicals produced from the substrate during irradiation, this can lead to dose rate effects. Two different types of interactions occur.
As discussed 
in ~\cite{Wulkop1995141}, the presence of oxygen
increases the number of migration mechanisms for the radicals produced during irradiation.  This allows the radicals produced during irradiation
and that can create color centers which absorb light 
to migrate, find other radicals, 
and reform good chemical bonds.
However, as discussed in~\cite{bross19921199}, Oxygen also 
interacts with the radicals produced during irradiation in a way
that forms color centers.  As shown in~\cite{bross19921199}, 
initially after a high dose rate irradiation, the damage when
oxygen is present is less because of the increased migration.  However,
after allowing for an annealing (migration and annihilation of radicals, which can be encouraged 
through oxygen or heat) period after irradiation, 
the permanent damage is worse with oxygen~\cite{bross19921199,zorn2}
due to the extra color centers formed in its presence.
The amount of color centers that absorb at the 
wavelength emitted by the secondary dopant (permanent damage to
the absorption length) seems to be independent of the
dose rate \cite{sauli}.
However, as discussed in~\cite{bross19921199}, this may not be true
for the light emitted by the substrate, in the deep UV.


Simulations based on oxygen diffusion have been shown
to reproduce the time dependence of the induced attenuation length in scintillators based on PS and PMMA~\cite{Wick2001341}.  
Thus, immediately after irradiation at a high dose rate (Mrads/hr), 
the probability of absorption for visible light increases for wavelengths shorter than around 600 nm (yellow) due to color centers that form in the substrate~\cite{Bross199135}.  At low dose rate, because the oxygen penetrates deeper,
the initial damage is less.  At very low dose rates, sufficient migration
can occur even without oxygen, reducing the initial damage~\cite{zorn2}.


Studies on the effect of radiation on the light yield are fewer
than those on self-absorption.
Several studies indicate that the damage is to the substrate
and/or the transfer of its excitation to the dopants and not to the
dopants themselves.
In an early study, Rosman and Zimmer~\cite{rosmanzimmer} found that the light
output for organic scintillators based on PS versus dose is described,
approximately and for relatively large dose rates, by a double exponential, 
with a small-dose component whose constant is approximately 100 Mrad.
For a PS scintillator doped with 1.5\% TPB, they also used UV light to illuminate
the samples  and found that this showed less
reduction of light than excitation via charged particles.
Since PS is transparent to UV, this
indicates that the TPB was not damaged, and instead the damage
was either to the PS or the migration of excitation from the PS
to the dopant.  Subsequent studies (\cite{berlman,173178,bros19921199}) came to a similar conclusions for
liquid scintillators and for various plastic scintillators respectively. 

If the damage is related to oxygen diffusion, a dose rate effect
is expected.
An interesting result, from 1996, is described in ~\cite{Biagtan1996125}.  They look at light output reduction for two
PS-based scintillators (SCSN-38 and SCSN-81, from Kuraray) and a
PVT-based scintillator (Bicron-499-35).  
Measurements were done as a function of time subsequent to irradiation, until the light output stabilized, using a ${\rm ^{241}AM}$ alpha source.
They find a reduction in post-recovery light output that depends linearly on the
log of the dose rate for dose rates ranging from $0.01$ to
$2$ Mrad/hr.  Note that the effect is not small: for a
dose of 2 Mrad, the light loss is negligible for a dose rate
of 2 Mrad/hr but 20\% for a dose rate of 0.01 Mrad/hr for SCSN-81.

Two studies may be in contradiction to those described above.
Two indicated very little damage to light output, and one indicated
damage to the dopant.

In ~\cite{Giokaris1993315}, from 1993, SCSN-81, SCSN-23, and 3HF (based on PS) were studied at doses between 10 krad to 1 Mrad and three dose rates, 1.2 Mrad/hr, 0.01 Mrad/h, using a long piece of plastic and a ${\rm ^{90}Sr}$ beta source.
They also looked at 0.0003 Mrad/hr with doses of 1.3 and 0.57 Mrad.
They saw a small decrease in light output after recovery, indicating
very little effect on the light yield.
In another study, by Bross, Pla-Dalmau, {\it et al}. from 1991~\cite{Bross199135}, the authors looked at light output for various 
primary (DAT, MOPOM, OLIGO408, OLIGO415A) and 
secondary (BBQ, K27, DMPOPOP, 3HF) dopants 
in PS
for a dose of 10 Mrad accumulated with a relatively high  dose rate (0.4 Mrad/h).  They looked at this for 2 different concentrations of the secondary dopant.  They optimized the concentration of the first dopant using the light output before irradiation.  They found no change to the intrinsic light output, indicating at high dose rates the dopants were
not damaged by this large dose.  
Perhaps the damage was not seen in the first due to the relatively low doses and in the second due to the high dose rate.


In ~\cite{Wick1991472}, decreased light
output for a plastic scintillator from Kuraray, SCSN-38, which
is based on PS with b-PBD primary and BDB secondary dopants, was seen.
The main absorption wavelengths for PS typically range between
between 230 to 260 nm, for the primary dopant b-PBD between 270 to 330 nm,
and for the wavelength shifting dopant BDB between 310 to 400 nm.
They found that light loss was much stronger in the presence of oxygen.
Note that while oxygen plays a beneficial role in regards
to annealing of induced absorption length at the end of radiation, 
it plays a detrimental role in 
regards to light output.  
They also found the light output loss
was independent of the wavelength of the light
that was used to excite the scintillator when it
was varied between 230 and 400 nm.
From this they conclude that the damage is due to destruction
of the second flour.  This is different than what was found in
the other studies, which indicated that damage to the dopants was
small and that damage was mostly to the substrate,
although different dopants were used in the studies.
By looking at the damage as a function of the thickness of the scintillator,
they concluded that the BDB molecules are mainly destroyed
near the surface, which lends support to a mechanism involving
oxygen diffusion.


If there is damage, and the damage is to the transfer mechanism, increasing the dopant
concentration could help.
In ~\cite{zorn3} (perhaps better documented in ~\cite{sauli}), the authors show that for a PS fiber with primary dopant of PTP and secondary dopant of 3HF,
increasing the dopant concentration from nominal to 20 times nominal continuously increases the light output after 100 Mrad (dose rate unknown).

In ~\cite{Majewski1989500}, the radiation resistance of BC-408 (from Bicron Corporation) was studied for different concentrations of the dopant from half to $3/2$ the nominal concentration.  The study was done at very high dose rate (36 Mrad/hr) and a total dose of 3 Mrad.  They saw that varying the concentration of the secondary fluor did not affect the output as studied with a $\rm{^{207}Bi}$ electron source.  They found decreasing the secondary dopant made it less rad hard, but increasing did not help.  
Also, based on studies comparing PVX to PS and PVT, the authors of \cite{barashkov19961557} also concluded that
increasing the concentration of the primary dopant might reduce radiation damage.



In order to understand the relative role of destruction of the dopants versus damage to the substrate for modern scintillating plastics,
we have studied the light output for two plastic scintillators, EJ-200 and EJ-260, varying the concentration of the dopants, for
different total doses and dose rates.  




\section{Measurements}
Both EJ-200 and EJ-260 use PVT as a substrate.  EJ-200 has a light output that is 60\% of anthracene and a wavelength of maximum emission of 435 nm (blue).  
EJ-260 also has a light output that is 60\% of anthracene but a wavelength of maximum emission of 490 nm (green).  Eljen prepared scintillator bars
with dimensions of 1x1x5cm.  For the EJ-200, bars were made with concentration of the primary scintillation dopant at 0.5, 1.0, 1.5, and 2.0 that of
the standard concentration.  For the EJ-260, bars with made with concentrations of the primary (x) and secondary (p) dopant of 1x1p, 1x2p,1x4p,2x1p,4x1p,2x2p, and 4x4p.
The fluorescence output was measured using a fluoromax-4 fluorometer by Horiba Scientific using a right-angle configuration and an excitation wavelength of
{\color{red} xx} nm.  For the absorption measurements, a {\color{red} xx} by {\color{red} xx} was used.
Air was used as the reference.  In ~\cite{Bross199135}, the integration
of the fluorescence spectra was shown to reproduce the results using a
$\rm{^{207}Bi}$ source to within a few percent for most dopants.

Radiations were done using a ${\rm ^{60}Co}$ source at the University of Maryland with an activity of {\color{red} xxx}.  
The dose was measured using {\color{red} i don't know}

Figure~\ref{fig:ej200doping1x} shows the spectra for EJ-200 with nominal doping before irradiation and after
30 Mrad at 1 Mrad/hr and after 50 Mrad at 1 Mrad/hr.

\begin{figure}[!ht]
\begin{center}
\includegraphics[width=0.49\textwidth]{./figures/placeholder.pdf}
\caption{
Emission spectrum for EJ-200 at nominal doping before irradiation, after 30 Mrad at 1 Mrad/hr and after 50 Mrad at 1 Mrad/hr.
}
\label{fig:ej200doping1x}
\end{center}
\end{figure}

%%%%%%%%%%%%%%%%%%%%%%%%%%%%%%%%%%%%
%%%%%%%%%%%%%%%%%%%%%%%%%%%%%%%%%%%%


Table~\ref{tab:ResultsEJ200} shows the ratio of the light output
at {\color{red} xxx nm}
for the EJ-200  to that before irradiation for different doses, dose rates, and dopant concentrations.
Figure~\ref{fig:fig2} shows the ratio of the light output
at {\color{red} xxx nm}
for the EJ-200 to that before irradiation as a function of dose rate
for several total doses.

\begin{table}[thb]
\centering
\caption{
Ratio of average light output at {\color{red} xxx nm} to that before irradiation
for EJ-200 scintillator.  For the concentrations
of the dopants, NpMs refers to N times the default concentration of primary and M times the default concentration of the secondary dopant
the commercial version of the product.
}
\label{tab:ResultsEJ200}
{\small
\begin{tabular}{|l| l| l| l| l| l|}
\hline
Dose & Dose rate  &\multicolumn{4}{|l|}{light output ratio for dopings} \\
(Mrad)  & (Mrad/hr) & 1p1s & 0.5p1s & 1.5p1s & 2.0p1s   \\
\hline
\hline
0 & $-$ & 1.0 & {\color{red} xx} & {\color{red} xx} & {\color{red} xx}\\ \hline
2 & 0.1 & {\color{red} xx} & {\color{red} xx} & {\color{red} xx} & {\color{red} xx}\\
2 & 0.3 & {\color{red} xx} & {\color{red} xx} & {\color{red} xx} & {\color{red} xx}\\
2 & 1.0 & {\color{red} xx} & {\color{red} xx} & {\color{red} xx} & {\color{red} xx}\\ \hline
4 & 0.1 & {\color{red} xx} & {\color{red} xx} & {\color{red} xx} & {\color{red} xx}\\
4 & 0.3 & {\color{red} xx} & {\color{red} xx} & {\color{red} xx} & {\color{red} xx}\\
4 & 1.0 & {\color{red} xx} & {\color{red} xx} & {\color{red} xx} & {\color{red} xx}\\ \hline
10 & 0.3 & {\color{red} xx} & {\color{red} xx} & {\color{red} xx} & {\color{red} xx}\\
10 & 1.0 & {\color{red} xx} & {\color{red} xx} & {\color{red} xx} & {\color{red} xx}\\ \hline
30 & 1.0 & {\color{red} xx} & {\color{red} xx} & {\color{red} xx} & {\color{red} xx}\\ \hline
50 & 1.0 & {\color{red} xx} & {\color{red} xx} & {\color{red} xx} & {\color{red} xx}\\ \hline
\hline
\end{tabular}
}
\end{table}


\begin{figure}[!ht]
\begin{center}
\includegraphics[width=0.49\textwidth]{./figures/placeholder.pdf}
\caption{
Ratio average light output for EJ-260 at {\color{red} xxx nm} to that before irradiation versus dose rate for several total doses.
}
\label{fig:fig2}
\end{center}
\end{figure}

%%%%%
%%%%%
%%%%%
Table ~\ref{tab:ResultsEJ260} shows the ratio of the light output at {\color{red} xxx nm}
for EJ-260 to that before irradiation for different doses, dose rates, and
dopant concentrations.
Figure~\ref{fig:fig3} shows the ratio of the light output at {\color{red} xxx nm}
for EJ-260 to that before irradiation as a function of dose rate
for several total doses.

\begin{table}[thb]
\centering
\caption{
Ratio of light output at {\color{red} xxx nm} to that before irradiation
for EJ-260 scintillator.  For the concentrations
of the dopants, NpMs refers to N times the default concentration of primary and M times the default concentration of the secondary dopant
the commercial version of the product.
}
\label{tab:ResultsEJ260}
{\small
\begin{tabular}{|l| l| l| l| l| l| l| l| l|}
\hline
Dose & Dose rate  &\multicolumn{7}{|l|}{light output ratio for dopings} \\
(Mrad)  & (Mrad/hr) & 1p1s & 2p1s & 4p1s & 1p2s & 1p4s & 2p2s & 4p4s   \\
\hline
\hline
0 & $-$ & 1.0 & {\color{red} xx} & {\color{red} xx} & {\color{red} xx}& {\color{red} xx} & {\color{red} xx} & {\color{red} xx}\\ \hline
2 & 0.1 & {\color{red} xx} & {\color{red} xx} & {\color{red} xx} & {\color{red} xx}& {\color{red} xx} & {\color{red} xx} & {\color{red} xx}\\
2 & 0.3 & {\color{red} xx} & {\color{red} xx} & {\color{red} xx} & {\color{red} xx}& {\color{red} xx} & {\color{red} xx} & {\color{red} xx}\\
2 & 1.0 & {\color{red} xx} & {\color{red} xx} & {\color{red} xx} & {\color{red} xx}& {\color{red} xx} & {\color{red} xx} & {\color{red} xx}\\ \hline
4 & 0.1 & {\color{red} xx} & {\color{red} xx} & {\color{red} xx} & {\color{red} xx}& {\color{red} xx} & {\color{red} xx} & {\color{red} xx}\\
4 & 0.3 & {\color{red} xx} & {\color{red} xx} & {\color{red} xx} & {\color{red} xx} & {\color{red} xx} & {\color{red} xx} & {\color{red} xx}\\ 
4 & 1.0 & {\color{red} xx} & {\color{red} xx} & {\color{red} xx} & {\color{red} xx}& {\color{red} xx} & {\color{red} xx} & {\color{red} xx}\\ \hline
10 & 0.3 & {\color{red} xx} & {\color{red} xx} & {\color{red} xx} & {\color{red} xx}& {\color{red} xx} & {\color{red} xx} & {\color{red} xx}\\
10 & 1.0 & {\color{red} xx} & {\color{red} xx} & {\color{red} xx} & {\color{red} xx}& {\color{red} xx} & {\color{red} xx} & {\color{red} xx}\\ \hline
30 & 1.0 & {\color{red} xx} & {\color{red} xx} & {\color{red} xx} & {\color{red} xx}& {\color{red} xx} & {\color{red} xx} & {\color{red} xx}\\ \hline
50 & 1.0 & {\color{red} xx} & {\color{red} xx} & {\color{red} xx} & {\color{red} xx}& {\color{red} xx} & {\color{red} xx} & {\color{red} xx}\\ \hline
\hline
\end{tabular}
}
\end{table}


\begin{figure}[!ht]
\begin{center}
\includegraphics[width=0.49\textwidth]{./figures/placeholder.pdf}
\caption{
Ratio average light output for EJ-260 at {\color{red} xxx nm} to that before irradiation versus dose rate for several total doses.
}
\label{fig:fig3}
\end{center}
\end{figure}



\section{Interpretation}


If this dose rate effect is due to oxygen diffusion, then
we expect it to be governed by the diffusion equation.  Since
the oxygen diffusion depth is greater at lower dose rates,
we expect the damage for the same dose at different dose rates to
increase up to the point where the dose rate is low enough that oxygen
permeates the entire sample for particles such as minimum ionizing
particles that transverse the sample.  For particles such as alphas
and low energy $\beta$s, the saturation should occur sooner.
The dose rate effect should plateau
at this point.  Specifically, we predict (following \cite{Wick1991472}) that the light output
reduction should depend on both the dose and the dose rate as:
$$ L(R,D) = 1 - [f(D)Z(R) + a(D)(1-Z(R))]$$
where L is the \% light yield, D is the dose, R is the dose rate, 
Z(R) is the fraction of the scintillator containing 
oxygen and is given by
$$Z(R)=min({{2 z_0(R)}\over{d}},1)$$
where d is the thickness of the scintillator and $z_0$ gives
the depth of scintillator penetrated by oxygen,
a(D) is related to the fraction of quenching in the part of
the scintillator containing oxygen and is given by $1-e^{-a_0 D}$, 
and
f(D) is related to the fraction of quenching in the part of
the scintillator not containing oxygen and is given by $1-e^{-f_0 D}$.
The diffusion depth $z_0(R)$ is given by diffusion theory as
$$z_0(R)=\sqrt{\gamma/R}$$
where $\gamma$ is a property of the substrate and the radicals that
are dissolved in that substrate during the diffusion process.
The value for $\gamma$ depends on the material, oxygen pressure,
radical concentration (which is proportional to the dose) and 
temperature.
The minimization in Z(R) occurs because the fraction of scintillator
containing oxygen can not exceed one.  Note that this equation 
assumes the temperature and the surrounding atmosphere
is held constant, as this can 
affect diffusion.



\begin{figure}[!ht]
\begin{center}
\includegraphics[width=0.49\textwidth]{./figures/Biagtan.pdf}
\caption{
Comparison of results with data from Biagtan {\it et al.}~\cite{Biagtan1996125}
}
\label{fig:ej200doping1x}
\end{center}
\end{figure}



\section{Conclusions}

\section{Acknowledgments}
The authors would like to thank Neil Blough and Anna Pla-Dalmau for advice on taking fluorescence measurements, and Blough for the use of his machine. We thank Saint Anselm's Abbey school for use of their 3D printer.   
The authors would like to thank {\color{red} various people} at
the University of Maryland's Nuclear Reactor and Radiation
Facilities group for assistance
with the irradiations.
This work was supported in part by U.S. Department of Energy Grant DESC0010072.

\section*{References}

\bibliography{afluorescence}

\end{document}
