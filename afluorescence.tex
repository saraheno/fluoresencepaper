\documentclass[review]{elsarticle}

\usepackage{lineno,hyperref}
\modulolinenumbers[5]

\journal{Nuclear Instruments and Methods B}

%%%%%%%%%%%%%%%%%%%%%%%
%% Elsevier bibliography styles
%%%%%%%%%%%%%%%%%%%%%%%
%% To change the style, put a % in front of the second line of the current style and
%% remove the % from the second line of the style you would like to use.
%%%%%%%%%%%%%%%%%%%%%%%

%% Numbered
%\bibliographystyle{model1-num-names}

%% Numbered without titles
%\bibliographystyle{model1a-num-names}

%% Harvard
%\bibliographystyle{model2-names.bst}\biboptions{authoryear}

%% Vancouver numbered
%\usepackage{numcompress}\bibliographystyle{model3-num-names}

%% Vancouver name/year
%\usepackage{numcompress}\bibliographystyle{model4-names}\biboptions{authoryear}

%% APA style
%\bibliographystyle{model5-names}\biboptions{authoryear}

%% AMA style
%\usepackage{numcompress}\bibliographystyle{model6-num-names}

%% `Elsevier LaTeX' style
\bibliographystyle{elsarticle-num}
%%%%%%%%%%%%%%%%%%%%%%%

\begin{document}

\begin{frontmatter}

\title{Fluorescence yield for plastic scintillators after radiation }


%% or include affiliations in footnotes:
\author[umd]{Zishuo Yang\corref{mycorrespondingauthor}}
\cortext[mycorrespondingauthor]{Corresponding author}
\ead{yangzs@terpmail.umd.edu}
\author[umd]{Alberto Belloni}
\author[umd]{Sarah C. Eno}
\author[eljen]{Charles Hurlbut}
\author[umd]{Geng Yuan Jeng}
\author[umd]{Yao Yao}




\address[umd]{Dept. Physics, U. Maryland, College Park MD 30742 USA}
\address[eljen]{Eljen Technology, 1300 W. Broadway, Sweetwater, Tx 79556 USA}


\begin{abstract}
The fluorescence yield versus wavelength for plastic scintillators EJ-408 and EJ-260 for doses of 50, 30, 10, 4, and 2 Mrad from a $\rm {^{60}Co}$ source at various dose rates and for different concentrations of the primary and secondary dopant.  While the nominal dopant concentration gives the highest light output prior to irradiation, a higher concentration is found to be optimal for irradiated plastics.
\end{abstract}

\begin{keyword}
\texttt{elsarticle.cls}\sep plastic scintillator\sep fluorescence\sep radiation hardness\sep 
\end{keyword}

\end{frontmatter}

\linenumbers

\section{Introduction}
Organic scintillators (such as toluene, polystyrene, and naphthalene) containing wave-length shifting
additives in solution have long been popular elements in detectors used
in particle physics, nuclear physics, radiation safety, and heath physics applications  due to their high light output, low cost, fast response,
and versatility of physical construction. 
Plastic scintillators and wavelength shifters, including
wavelength shifting fibers, are currently available
from companies such as St. Gobain \cite{stgobain}, Kuraray \cite{kuraray}, and 
Eljen \cite{eljen}.
Prolonged exposure of plastic scintillator to
ionizing radiation has harmful effects: it can increase
light self-absorption (yellowing) and  decrease
initial light yield.  




\section{Results}


\section{Conclusions}

\section{Acknowledgements}
The authors would like to thank Neil Blough and Anna Pla-Dalmau for advice on taking fluorescence measurements, and Blough for the use of his machine.  This work was supported in part by U.S. Department of Energy Grant YYYYY.

\section*{References}

\bibliography{afluorescence}

\end{document}